\chapter{Abstraktion und Formalismus}
\label{sec:AuF}
%Kerngedanke der Mathemathik
%Einführung in die höhere Mathematik
%neue Arbeitsweisen
Wir werden einige aus dem Schulunterricht unbekannte mathematische Arbeitsweisen und den Kerngedanken der höheren Mathematik erlernen.

\section{mathematische Sprache}
Eine gängige Strukturierung eines mathematischen Textes ist durch die Unterteilung in verschiedene Abschnitte wie folgt gegeben:

\theoremstyle{definition}
\begin{definition}{}
Neue mathematische Begriffe und Sachverhalte werden \emph{definiert}, was bedeutet, dass sie axiomatisch eingeführt werden, also nicht bewiesen werden.
\end{definition}

\begin{proof}
Beweise sind im folgenden Stil aufgebaut:
\\ Vor.: In den \emph{Voraussetzungen} stehen alle für den Beweis notwendigen mathematischen Tatsachen.
\\ Beh.: Die \emph{Behauptung} verdeutlicht die zu beweisende/zeigende Tatsache.
\\ Bew.: Hier erfolgt der tatsächliche \emph{Beweis}. Dieser liefert ein Ergebnis.
\end{proof}

\theoremstyle{Corollar}
\begin{Corollar}
Ein \emph{Corollar} ist ein direkt aus einem der hier aufgeführten Begriffe abgeleitetes Ergebnis, das nicht zwangsweise einen Beweis erfordert.
\end{Corollar}

\theoremstyle{Lemma}
\begin{Lemma}{}
Ein \emph{Lemma} oder Hilfssatz ist als ein Ergebnis, das nur für weitere Beweisführungen wichtig sind, zu verstehen.
\end{Lemma}

\theoremstyle{Satz}
\begin{Satz}{}
Ein \emph{Satz} ist ein Ergebnis. 
\end{Satz}

\theoremstyle{theo}
\begin{theo}
Die Definition dieser Begrifflichkeit ähnelt der des Satzes, jedoch ist ein \emph{Theorem} von äußerst großer Wichtigkeit.
\end{theo}

\theoremstyle{prop}
\begin{prop}{}
Eine \emph{Proposition} oder Vorschlag wird in dieser Arbeit als eine aus einem Satz vermuteten Aussage benutzt, was dennoch einen Beweis benötigt.
\end{prop}

\theoremstyle{example}
\begin{example}
\emph{Beispiele} dienen der Vermittlung von einem intuitiven Verständnis abstrakter mathematischer Sachverhalte.
\end{example}

%\subsection*{Vokabelliste}

\section{Der Kerngedanke der Mathematik}
\begin{quote}
"Die Kunst, Mathematik zu machen, besteht darin, diesen speziellen Fall zu finden, der alle Elemente der Verallgemeinerung enthält."
\begin{flushright}
\textsc{David Hilbert}
\end{flushright}
\end{quote}

Die Mathematik will nicht für jeden Fall eigens eine Erkläruneg liefern, sie will alle Besonderheiten verallgemeinern. In dieser Arbeit zum Beispiel gilt es eine Aussage nicht nur für den Spezialfall von einer oder zweier Dimensionen zu beweisen, sondern sie bezieht sich gleich auf alle Dimensionen. 
\\Jedoch werden verallgemeinernde Behauptungen sehr abstrakt, was beim einmaligen Durchlesen oftmals große Schwierigkeiten bereiten wird. Um dies vorzubeugen, werden viele Beispiele und Kommentare zu sehr schwierig vorstellbaren Inhalten wie denen, die höhere Dimensionen betreffen, geliefert.


\section{Stufen der Abstraktion}
\subsection*{Teil I}
Der erste Teil dient der Einführung des abstrakten Denkens durch viele visualisierende Beispiele und Modelle. Hier werden wir Würfel in verschiedenen Dimension behandeln.
\\ Der Abstrakitionsgehalt hält sich auf einem einfacheren Niveau.
\subsection*{Teil II}
Die erforderlichen mathematischen Kenntnisse für die Definition des Dimensionenbegriffs werden erklärt. Es werden Räume endlicher (oder $n$-ter) Dimension betrachtet. Die Beispiele befinden sich noch innnerhalb der Grenzen der Vorstellbarkeit.
\\ Die Stufe der Abstraktion befindet sich auf einer mittleren Ebene.
\subsection*{Teil III}
Im abstraktesten Teil behandeln wir unendliche Dimensionen. Wir bewegen uns außerhalb des Vorstellbaren und müssen uns daher formal\footnote{Wir berufen uns auf Definitionen aus dem zweiten Teil, die uns bestimmte Eigenschaften über höhere Dimensionen vorgeben.} unendliche Dimensionen vorstellen.



