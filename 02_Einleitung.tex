\chapter{Abstraktion und Formalismus}
\label{sec:AuF}
%Kerngedanke der Mathemathik
%Einführung in die höhere Mathematik
%neue Arbeitsweisen
%Wir werden einige aus dem Schulunterricht unbekannte mathematische Arbeitsweisen und den Kerngedanken der höheren Mathematik erlernen.
Diese Arbeit beschäftigt sich mit Räumen und geometrischen Figuren $n$-ter Dimension.\footnote{Es existieren auch Räume von irrationaler Dimensionenzahl (vgl. \cite{Fraktale}).} Die Variable $n$ steht immer für $n\in \mathbb{N}_0$.
\section{Der Kerngedanke der Mathematik}
\begin{quote}
"Die Kunst, Mathematik zu machen, besteht darin, diesen speziellen Fall zu finden, der alle Elemente der Verallgemeinerung enthält."
\begin{flushright}
\textsc{David Hilbert}
\end{flushright}
\end{quote}

Die Mathematik will nicht für jeden Fall eigens eine Erklärung liefern, sie will alle Besonderheiten verallgemeinern. In dieser Arbeit zum Beispiel gilt es, eine Aussage nicht nur für den Spezialfall von einer oder zweier Dimensionen zu beweisen, sondern sie bezieht sich gleich auf alle Dimensionen. 
Jedoch werden verallgemeinernde Behauptungen sehr abstrakt, was oftmals große Schwierigkeiten bereiten wird. Um dem vorzubeugen, werden viele Beispiele, Modelle und Kommentare zu sehr schwierig vorstellbaren Inhalten wie denen, die höhere Dimensionen betreffen, geliefert.
%%%%%%%%%%%%%%%%%%%%%%%%%%%%%%%%%%%%%%%%%%%%%%%%%%%%%%%%%%%%%%%%%%%%%%%%%%%%%%%%%%%%%%%%%%%%%%%%%%%%%%%%%%%%%%%%%%%%%%%%%%%%%%%%%%%%%%%%%%%%%%%%%%%%%%%%%%%
\newpage
\section{Inhalt}
 Wir wollen die höheren Dimensionen zunächst visuell, dann mathematisch in der Reihenfolge ihrer aufsteigenden Schwierigkeit betrachten. 
\subsection*{Kapitel 2}
Hier werden wir Koordinatensysteme und Würfel in $n$ Dimension anhand von Gedankenexperimenten behandeln. Diese Sachverhalte kann man sich noch einfach vorstellen.
\subsection*{Kapitel 3}
Die erforderlichen mathematischen Kenntnisse aus der Linearen Algebra für die Definition des endlichen Dimensionenbegriffs werden erklärt. Wir werden lernen, sich höhere Dimension \emph{formal}, d.h. auf Papier mittels Formeln, vorzustellen.
\subsection*{Kapitel 4}
Im abstraktesten Teil behandeln wir unendliche Dimensionen, wobei zwischen den verschiedenen Unendlichkeiten unterschieden wird.
%%%%%%%%%%%%%%%%%%%%%%%%%%%%%%%%%%%%%%%%%%%%%%%%%%%%%%%%%%%%%%%%%%%%%%%%%%%%%%%%%%%%%%%%%%%%%%%%%%%%%%%%%%%%%%%%%%%%%%%%%%%%%%%%%%%%%%%%%%%%%%%%%%%%%%%%%%%
\newpage
\section{Mathematische Sprache}
Eine gängige Strukturierung eines mathematischen Textes ist durch die Unterteilung in verschiedene Abschnitte wie folgt gegeben:

\theoremstyle{definition}
\begin{definition}{}
Neue mathematische Begriffe und Sachverhalte werden definiert, was bedeutet, dass sie axiomatisch eingeführt werden, also nicht bewiesen werden.
\end{definition}

\begin{proof}
Beweise sind im folgenden Stil aufgebaut:
\\ Vor.: In den Voraussetzungen stehen alle für den Beweis notwendigen mathematischen Tatsachen.
\\ Beh.: Die Behauptung verdeutlicht die zu beweisende/zeigende Tatsache.
\\ Bew.: Hier erfolgt der tatsächliche Beweis. Dieser liefert ein Ergebnis.
\end{proof}

\theoremstyle{Satz}
\begin{Satz}{}
Ein Satz ist ein Ergebnis. 
\end{Satz}

\theoremstyle{Lemma}
\begin{Lemma}{}
Ein Lemma oder Hilfssatz ist als ein Ergebnis zu verstehen, das nur für weitere Beweisführungen wichtig ist.
\end{Lemma}

\theoremstyle{theo}
\begin{theo}
Die Definition dieser Begrifflichkeit ähnelt der des Satzes, jedoch ist ein Theorem von äußerst großer Wichtigkeit.
\end{theo}

\theoremstyle{prop}
\begin{prop}{}
Eine Proposition oder Vorschlag wird in dieser Arbeit als eine aus einem Satz vermuteten Aussage benutzt, was dennoch einen Beweis benötigt.
\end{prop}

\theoremstyle{Corollar}
\begin{Corollar}
Ein Corollar ist ein direkt aus einem der hier aufgeführten Begriffe abgeleitetes Ergebnis, das nicht zwangsweise einen Beweis erfordert.
\end{Corollar}

\theoremstyle{example}
\begin{example}
Beispiele dienen der Vermittlung von eines Verständnisses abstrakter mathematischer Sachverhalte.
\end{example}

Hier sind einige wichtige Vokabeln und Symbole erklärt:
\begin{itemize}
\item Mit "$x\vcentcolon=y$" wird symbolisiert, dass $x$ als $y$ definiert wird, wobei die Variablen für einen beliebigen Ausdruck stehen.
\item "Abbildung" ist ein Synonym zu Funktion.
\end{itemize}
%\subsection*{Vokabelliste}







