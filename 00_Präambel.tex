%Dokumentklasse
\documentclass[a4paper,12pt,pointlessnumbers]{scrreprt}
\usepackage[left=3cm,right =25mm, bottom =25mm, top=25mm, bindingoffset=0mm]{geometry}
\usepackage[onehalfspacing]{setspace}
% ============= Packages =============

% Dokumentinformationen
\usepackage[
	pdftitle={Einblick in höhere Dimensionen},
	pdfsubject={},
	pdfauthor={Hanifah Muhammad},
	pdfkeywords={},	
	%Links nicht einrahmen
	hidelinks
]{hyperref}



% Standard Packages
\usepackage[utf8]{inputenc}
\usepackage[ngerman]{babel}
\usepackage[T1]{fontenc}
\usepackage{graphicx}
\graphicspath{{img/}}
\usepackage{fancyhdr}
\usepackage{lmodern}
\usepackage{color}
\usepackage{tkz-euclide}
\usepackage[nolist]{acronym} %Abkürzungen
\usepackage{sidecap}
\usepackage{subcaption}

% zusätzliche Schriftzeichen der American Mathematical Society
\usepackage{amsfonts}
\usepackage{amsmath}
	\DeclareMathOperator{\Abb}{Abb}
\usepackage{amsthm}
\usepackage{stmaryrd}
\usepackage{mathtools}

% Gänsefüßchen
\usepackage{csquotes}
\MakeOuterQuote{"}

%Für Vektorpraphiken
\usepackage{transparent}
\graphicspath{{Bilder/}}



%================== amsthm ====================
\theoremstyle{definition}
\newtheorem{definition}{Definition}[section]
\newtheorem{preremark}{Definition}
\newenvironment{Definition}{\begin{preremark}\upshape}{\end{preremark}}
\newtheorem{example}[definition]{Beispiel}
\newtheorem{bem}[definition]{Bemerkung}
\newtheorem{theo}[definition]{Theorem}
\newtheorem{Satz}[definition]{Satz}
\newtheorem{Corollar}[definition]{Corollar}
\newtheorem{prop}[definition]{Proposition}
\newtheorem{Lemma}[definition]{Lemma}



%nicht einrücken nach Absatz
%\setlength{\parindent}{0pt}


% ============= Kopf- und Fußzeile =============
\pagestyle{fancy}
%
\lhead{}
\chead{}
\rhead{\slshape \leftmark}
%%
\lfoot{}
\cfoot{\thepage}
\rfoot{}
%%
\renewcommand{\headrulewidth}{0.4pt}
\renewcommand{\footrulewidth}{0pt}
\renewcommand{\figurename}{Abb.}
% ============= Package Einstellungen & Sonstiges ============= 
%Besondere Trennungen
\hyphenation{De-zi-mal-tren-nung}

% ============= Modifikationen von Aufzählungen ===============
\renewcommand\theenumi{\roman{enumi}}
\renewcommand\labelenumi{(\theenumi)}




% ============= Dokumentbeginn =============

\begin{document}
%Seiten ohne Kopf- und Fußzeile sowie Seitenzahl
\pagestyle{plain}

\begin{center}
\begin{tabular}{p{\textwidth}}


\begin{center}
%\includegraphics[scale=0.5]{img/logos.jpg}
\end{center}







\begin{center}
\textbf{\Large{Abschlussarbeit}}
\end{center}


\begin{center}
zur Erlangung des akademischen Grades\\
Bachelor of Engineering
\end{center}


\begin{center}
vorgelegt von
\end{center}

\begin{center}
\large{\textbf{Max Mustermann}} \\
\small{geboren am 01.01.1900 in Musterhausen}
\end{center}

\begin{center}
\large{im Dezember 2014}
\end{center}

\\

\\

\begin{center}
\begin{tabular}{lll}
\textbf{Erstprüfer:} & & Prof. Dr. med. Dr.-Ing. M. Mustermann\\
\textbf{Zweitprüfer:} & &Prof. Dr.-Ing. F. Musterfrau\\
\end{tabular}
\end{center}

\end{tabular}
\end{center}

% Beendet eine Seite und erzwingt auf den nachfolgenden Seiten die Ausgabe aller Gleitobjekte (z.B. Abbildungen), die bislang definiert, aber noch nicht ausgegeben wurden. Dieser Befehl fügt, falls nötig, eine leere Seite ein, sodaß die nächste Seite nach den Gleitobjekten eine ungerade Seitennummer hat. 
\cleardoubleoddpage

% pagestyle für gesamtes Dokument aktivieren
\pagestyle{fancy}

%Inhaltsverzeichnis
\tableofcontents



\begin{acronym}
 \acro{VR}{Vektorraum}
 \acrodefplural{VR}[VRs]{Vektorräume}
 \acro{UVR}{Untervektorraum}
 \acrodefplural{UVR}[UVRs]{Untervektorräume}
 \acro{EZS}{Erzeugendensystem}
 \acrodefplural{EZS}[EZSs]{Erzeugendensysteme}
 \acro{Dim}{Dimension}
 \acrodefplural{Dim}[Dims]{Dimensionen}
 \acro{Linkomb}{Linearkombination}
 \acrodefplural{Linkomb}[Linkombs]{Linearkombinationen}
 \acro{linunab}{linear unabhängig}
 \acro{Linunab}{Lineare Unabhängigkeit}
 \acro{linab}{linear abhängig}
 \acro{Linab}{Lineare Abhängigkeit}
 \acro{R2}{$\mathbb{R}^2$}
 \acro{R3}{$\mathbb{R}^3$}
 \acro{vs}{v_1, ..., v_k}
 \acro{Summe}{\sum \limits_{i=1}^{n} \lambda_i v_i}
 \acro{ang}{Angenommen}
 \acro{dim}{dimensional}
 \acro{ndim}{$n$-dimensional}
 \acro{unenddim}{unendlichdimensional}
 \acro{Ev}{Einheitsvektor}
 \acrodefplural{Ev}[Evs]{Einheitsvektoren}
 \acro{PV}{Polynomvektorraum}
 \acro{RX}{\mathbb{R}[x]}
 \acro{BRX}{B_\mathbb{R}[x]}
 \acro{KoS}{Koordinatensystem}
 \acro{R2}{\mathbb{R}^2} 
\end{acronym}
\chapter{Abstraktion und Formalismus}
\label{sec:AuF}
%Kerngedanke der Mathemathik
%Einführung in die höhere Mathematik
%neue Arbeitsweisen
%Wir werden einige aus dem Schulunterricht unbekannte mathematische Arbeitsweisen und den Kerngedanken der höheren Mathematik erlernen.
Diese Arbeit beschäftigt sich mit Räumen und geometrischen Figuren $n$-ter Dimension.\footnote{Es existieren auch Räume von irrationaler Dimensionenzahl (vgl. \cite{Fraktale}).} Die Variable $n$ steht immer für $n\in \mathbb{N}_0$.
\section{Der Kerngedanke der Mathematik}
\begin{quote}
"Die Kunst, Mathematik zu machen, besteht darin, diesen speziellen Fall zu finden, der alle Elemente der Verallgemeinerung enthält."
\begin{flushright}
\textsc{David Hilbert}
\end{flushright}
\end{quote}

Die Mathematik will nicht für jeden Fall eigens eine Erklärung liefern, sie will alle Besonderheiten verallgemeinern. In dieser Arbeit zum Beispiel gilt es eine Aussage nicht nur für den Spezialfall von einer oder zweier Dimensionen zu beweisen, sondern sie bezieht sich gleich auf alle Dimensionen. 
Jedoch werden verallgemeinernde Behauptungen sehr abstrakt, was oftmals große Schwierigkeiten bereiten wird. Um dies vorzubeugen, werden viele Beispiele, Modelle und Kommentare zu sehr schwierig vorstellbaren Inhalten wie denen, die höhere Dimensionen betreffen, geliefert.
%%%%%%%%%%%%%%%%%%%%%%%%%%%%%%%%%%%%%%%%%%%%%%%%%%%%%%%%%%%%%%%%%%%%%%%%%%%%%%%%%%%%%%%%%%%%%%%%%%%%%%%%%%%%%%%%%%%%%%%%%%%%%%%%%%%%%%%%%%%%%%%%%%%%%%%%%%%
\newpage
\section{Inhalt}
 Wir wollen die höheren Dimensionen zunächst visuell, dann mathematisch in der Reihenfolge ihrer aufsteigenden Schwierigkeit betrachten. 
\subsection*{Kapitel 2}
Hier werden wir Koordinatensysteme und Würfel in $n$ Dimension anhand von Gedankenexperimenten behandeln. Diese Sachverhalte kann man sich noch einfach vorstellen.
\subsection*{Kapitel 3}
Die erforderlichen mathematischen Kenntnisse aus der Linearen Algebra für die Definition des endlichen Dimensionenbegriffs werden erklärt. Wir werden lernen, sich höhere Dimension \emph{formal}, d.h. auf Papier mittels Formeln, vorzustellen.
\subsection*{Kapitel 4}
Im abstraktesten Teil behandeln wir unendliche Dimensionen, wobei zwischen den verschiedenen Unendlichkeiten unterschieden wird.
%%%%%%%%%%%%%%%%%%%%%%%%%%%%%%%%%%%%%%%%%%%%%%%%%%%%%%%%%%%%%%%%%%%%%%%%%%%%%%%%%%%%%%%%%%%%%%%%%%%%%%%%%%%%%%%%%%%%%%%%%%%%%%%%%%%%%%%%%%%%%%%%%%%%%%%%%%%
\newpage
\section{Mathematische Sprache}
Eine gängige Strukturierung eines mathematischen Textes ist durch die Unterteilung in verschiedene Abschnitte wie folgt gegeben:

\theoremstyle{definition}
\begin{definition}{}
Neue mathematische Begriffe und Sachverhalte werden definiert, was bedeutet, dass sie axiomatisch eingeführt werden, also nicht bewiesen werden.
\end{definition}

\begin{proof}
Beweise sind im folgenden Stil aufgebaut:
\\ Vor.: In den Voraussetzungen stehen alle für den Beweis notwendigen mathematischen Tatsachen.
\\ Beh.: Die Behauptung verdeutlicht die zu beweisende/zeigende Tatsache.
\\ Bew.: Hier erfolgt der tatsächliche Beweis. Dieser liefert ein Ergebnis.
\end{proof}

\theoremstyle{Satz}
\begin{Satz}{}
Ein Satz ist ein Ergebnis. 
\end{Satz}

\theoremstyle{Lemma}
\begin{Lemma}{}
Ein Lemma oder Hilfssatz ist als ein Ergebnis zu verstehen, das nur für weitere Beweisführungen wichtig ist.
\end{Lemma}

\theoremstyle{theo}
\begin{theo}
Die Definition dieser Begrifflichkeit ähnelt der des Satzes, jedoch ist ein Theorem von äußerst großer Wichtigkeit.
\end{theo}

\theoremstyle{prop}
\begin{prop}{}
Eine Proposition oder Vorschlag wird in dieser Arbeit als eine aus einem Satz vermuteten Aussage benutzt, was dennoch einen Beweis benötigt.
\end{prop}

\theoremstyle{Corollar}
\begin{Corollar}
Ein Corollar ist ein direkt aus einem der hier aufgeführten Begriffe abgeleitetes Ergebnis, das nicht zwangsweise einen Beweis erfordert.
\end{Corollar}

\theoremstyle{example}
\begin{example}
Beispiele dienen der Vermittlung von einem intuitiven Verständnis abstrakter mathematischer Sachverhalte.
\end{example}

Hier sind einige wichtige Vokabeln und Symbole erklärt:
\begin{itemize}
\item Mit "$x\vcentcolon=y$" wird symbolisiert, dass $x$ als $y$ definiert wird, wobei die Variablen für einen beliebigen Ausdruck stehen.
\item "Abbildung" ist ein Synonym zu Funktion.
\end{itemize}
%\subsection*{Vokabelliste}








\chapter{Visualaisierung höherer Dimensionen}
\label{sec:Visualisierung}

\section{Koordinatensysteme}

Um ein "Bild" von der vierten räumlichen Dimension\footnote{Wir sprechen hier nicht wie in der allgemeinen Relativitätstheorie von der vierten Dimension als die Zeit.} zu bekommen, machen wir ein kleines Gedankenexperiment: Wir beginnen mit einem nulldimensionalen Punkt. Dieser wird im nächsten Schritt unendlich oft in die erste Dimension, also nach rechts und links, verschoben, sodass eine eindimensionale Gerade entsteht. Diese soll den Zahlenstrahl $\mathbb{R}$ darstellen. Anschließend stellen wir uns eine Gerade vor, die senkrecht zu der reelen Zahlengeraden steht. Wir erhalten das zweidimensionale kartesische Koordinatensystem $\mathbb{R}^2$ (vgl. \ref{KoS}). Fügen wir eine dritte  Gerade, die zu den letzteren orthogonal steht, hinzu, erhalten wir das dreidimensionale Koordinatensystem $\mathbb{R}^3$ (vgl. \ref{KoS}). Führen wir den Gedanken ($n$ mal) fort, bekommen wir den $\mathbb{R}^4$ bzw. den $\mathbb{R}^n$ (siehe \ref{exampleVR}), die wir uns allerdings nicht vorstellen können. Unser Gehirn vermag keine Vorstellung der vierten Richtung, die als "ana und kata" bezeichnet wird.\footnote{Uns sind oben, unten, rechts, links, vorne und hinten bekannt, was alle Richtungen unserer dreidimensionale Welt beschreibt.}   %Quelle Michio S.127 also ana und kata
\begin{quote}
"Ein Mensch, der seine Existenz dem widmet, schafft es vielleicht, sich die vierte Dimension auszumalen."
\begin{flushright}
\textsc{Henri Poincaré}
\end{flushright}
\end{quote}
Für die $n$-te Richtung kann sich der Leser eigens einen Namen definieren. In Kapitel \ref{sec:VR} und \ref{sec:unendliche VRs} werden wir lernen, wie wir die visuelle Vorstellung durch eine mathematische "formale" Interpretation ersetzen.

\section{Konstruktion eines Hyperwürfels}
Wir wollen im Folgenden einen $n$-dimensionalen für Würfel $n\in \mathbb{N}_0$ bauen.
\begin{definition} {Hyperwürfel}
\\ Es sei ein $(n-1)$-dimensionaler Würfel. Dieser wird in die $n$-te (nächsthöhere) Dimension um eine Längeneinheit verschoben. Die gleichen Ecken werden miteinander verbunden. So erhalten wir einen $n$-dimensionalen Würfel.
\end{definition}

Wir beginnen mit einem nulldimensionalen Würfel, einem Punkt.
Anschließend verschieben wir ihn um eine Längeneinheit in die Länge, die erste Dimension, und verbinden die beiden Endpunkte. Nun erhalten wir eine Linie mit zwei Ecken. 
Danach wird sie in die Breite, die zweite Dimension, verschoben, sodass wir ein Quadrat nach der Verbindung der Eckpunkte bekommen. Verschiebt man im nächsten Schritt das Quadrat in die Tiefe, die dritte Dimension, und fügen die gleichen Ecken zusammen, wird es zu einem Würfel. Machen wir analog weiter, indem wir den Würfel nach "ana" also in die vierte Dimension verschieben und die Ecken verknüpfen,  entsteht ein \emph{Tesserakt} (siehe Modell 1). Natürlich lässt sich diese Konstruktion ("nach den Gesetzen der Analogie") beliebig oft weiterführen bis man schließlich einen $n$-dimensionalen Hyperwürfel erhält. 
\begin{figure}[h]
\centering
\input{Bilder/Wuerfel.pdf_tex}
\caption{Quelle: Wikipedia Tesserakt}
\label{Wuerfel}
\end{figure}

\indent Wir wollen nun die Anzahl der Ecken ermitteln: Der nulldimensionale Würfel besteht aus einem Eck, der eindimensionale aus zwei, der dreidimensionale aus 8. "1, 2, 4, [8] stellen offensichtlich eine geometrische Reihe dar." (Quelle: Flatland S.88) Somit hat der Tesserakt 16 und der $n$-dimensionale Würfel $2^n$ Ecken. 
\\ \indent Ein $n$-dimensionaler Würfel wird von $2n$ $(n-1)$-Würfeln begrenzt. Der nulldimensionale Würfel hat keine Grenzwürfel. Die Linie hat zwei Punkte als Enden. Die Grenzflächen des Quadrats sind die vier Seiten. Der Würfel wird er von sechs Quadraten begrenzt. "0, 2, 4, [6]" bilden eine "arithmetische" Reihe. Analog lässt sich berechnen, dass der vierdimensionale Würfel aus acht normalen Würfel besteht.\footnote{Diese können in Modell 1 und 4 nachgezählt werden.} Deshalb wird der Tesserakt auch als Octachoron (\textit{griech.} Achtzeller) (vgl. Ingo und Matthias Vortrag) bezeichnet.



\subsection*{Der Tesserakt}
	\subsubsection*{Faltung}
		 Die Faltung eines Tesserakts erfolgt analog wie die eines Würfels. Um sich das zu visualisieren, werden die Modelle 2 (zweidimenaionales Würfelnetz eines Würfels) und 3 (dreidimenionales Würfelnetz eines Octachoron) zur Verfügung gestellt. Sowie man einen Würfel erhält, indem man alle roten Quadrate des Modells 2 nach oben faltet, sodass sich das blaue direkt gegenüber dem gelben befindet, wäre es möglich die roten Würfel von Modell 3 nach "ana" zu falten, sodass der blaue gegenüber dem gelben\footnote{Ein roter Würfel lässt sich entfernen, damit man den innenliegenden gelben sehen kann.} läge. Im Anhang sieht man Animationen der Faltung eines Würfels und eines Tesserakts.
\subsubsection*{4d-Perspektive}
Es gibt auch eine andere dreidimensionale Darstellungweise als bei Modell 1, nämlich mit 4d-Perspektive. Sie funktioniert genauso wie die 3d-Perspektive. Weiter vom Betrachter entfernte Objekte werden kleiner gezeichnet. In Modell 4 ist der kleine Würfel im Inneren weiter in "ana"-Richtung entfernt. %Zitat Matts video



%\chapter{Algebraische Strukturen}
\label{sec:Algebraische Strukturen}
Wir werden uns mit der fundamentalsten aller algebraischen Strukturen, den Gruppen, befassen, um mit einem Zwischenstopp bei den Körpern die sogenannten Vektorräume über einen Körper $K$ zu definieren.
\section{Gruppen}
\label{sec:Gruppen}
Gruppen ermöglichen eine Abstrahierung von Rechenoperationen. Ebenso muss diese algebraische Struktur bestimmte Eigenschaften erfüllen, die im Folgenden nach einigen grundlegenden Definitionen neuer Begriffe aufgeführt werden.

\theoremstyle{definition}
\begin{definition}{\textbf{Kartesisches Produkt}}
	\\{\glqq}Das kartesische Produkt $A\times B$ zweier Mengen $A$ und $B$ ist die Menge aller geordneten Paare $(a,b)$ mit $a \in A$ und $b \in B$: \[A \times B := \{(a,b) \mid a \in A, b \in B\} \text{.{\grqq} \cite[S. 28]{Enzy}}\]
\end{definition}

%Fußnote, die auf Beispiel bei den VRs hinweist
\theoremstyle{bem}
\begin{bem}{}
Es kommt auf die Reihenfolge innerhalb des Paares an. Somit gilt $(a,b) \not= (b,a)$.
\end{bem}
In Abschnitt \ref{exampleVR} werden weitere Beispiele für das kartesische Produkt aufgeführt.

\theoremstyle{definition}
\begin{definition}{\textbf{innere Verknüpfung}}
	\\{\glqq}Eine (innere) Verknüpfung auf einer Menge $G$ ist eine Abbildung[\footnote{Das Wort Abbildung ist ein Synonym zu Funktion.}]
\begin{center}
$\mu: G \times G \rightarrow G $ .{\grqq} \cite[S. 19, 4.1]{Skript}
\end{center}
\end{definition}

\theoremstyle{bem}
\begin{bem}{}
Aus der Abbildungsvorschrift geht hervor, dass einem Paar $(g_1,g_2)$ ein Element $\mu((g_1,g_1))$, {\textendash} stattdessen schreiben wir auch $g_1 \, \cdot \, g_2$, $g_1 \, + \, g_2$ oder $g_1 \, \circ \, g_2$ {\textendash}, aus der Zielmenge zugeordnet wird (vgl. \cite[S. 19, 4.1]{Skript}).
\end{bem}

\theoremstyle {definition}
\begin{definition} {\textbf{Gruppe}
	\\ {\glqq}Eine Gruppe ist eine Menge $G$ mit einer Verknüpfung 
\begin{center}
$\circ$[\footnote{Die innere Verknüpfung wird als \glqq $\circ$\grqq \, bezeichnet. Anstatt des Zeichens kann eine Rechenoperation wie $+$ oder $\cdot$ verwendet werden.}]$ : G \times G \rightarrow G$ 
\end{center}
für die die folgenden Eigenschaften gelten
\begin{enumerate}
	\item (Assoziativität) Für alle $x, y, z \in G$ gilt 	
	\label{(i)}
	\[(x \circ y) \circ z = x \circ (y \circ z)\text{.}\]
	\item (Existenz eines neutralen Elements) Es gibt ein $e \in G$ mit  
	\label{(ii)}	
	\[e \circ x = x = x \circ e \,\text{für alle}\, x \in G \text{.}\]
	\item (Existenz von Inversen) Sei $x \in G$. Dann gibt es ein $y \in G$ mit 
	\label{(iii)}	
	\[y \circ x = e = x \circ y \text{.{\grqq} \cite[S. 19, 4.2]{Skript}}\]
\end{enumerate}
}\end{definition}

\theoremstyle{definition}
\begin{definition}{\textbf{abelsche oder kommutative Gruppe}}
	\\Man bezeichnet eine Gruppe auch als kommutativ oder abelsch, wenn für alle $a,b \in G$ gilt
\[a \circ b = b \circ a\text{. \cite[S. 19, 4.3]{Skript}}\]
\end{definition}

Im weiteren Verlauf werden einige Beispiele für Gruppen aufgeführt.
\theoremstyle{example}
\begin{example}{}
Die Verknüpfung $+$ auf der Menge der natürlichen Zahlen $\mathbb{N}$ erfüllt die Eigenschaften 
(\ref{(i)}), (\ref{(ii)}) {\textendash} das neutrale Element ist hierbei die Zahl $0$ {\textendash} , aber nicht (\ref{(iii)}), weil das Inverse einer natürlichen Zahl $n$ die Lösung der Gleichung $n + x = 0 = x + n $ für $x \in \mathbb{N}$ ist, wobei $ x = -n$ ergibt, aber $-n \not\in \mathbb{N}$, sondern $-n \in \mathbb{Z}$. Somit ist die Verknüpfung $+$ auf der Menge der ganzen Zahlen, man schreibt auch $(\mathbb{Z}, +)$, eine Gruppe. Insbesondere ist sie \emph{kommutativ} oder \emph{abelsch}.
\end{example}

\theoremstyle{example}
\begin{example}{}
Ebenso bildet die Verknüpfung $\cdot$ (die Multiplikation) auf der Menge der rationalen Zahlen $\mathbb{Q}\setminus\{0\}$  eine \emph{kommutative} Gruppe. Das neutrale Element bezüglich der Addition ist die $1$, das auch als {\glqq}Einselement{\grqq} \cite[S.  20, 4.1]{Skript} bezeichnet wird. Die Null ist ausgeschlossen, weil sie kein Inverses hat, also eine Zahl aus $\mathbb{Q}$, die multipliziert mit Null das $1$ ergibt, existiert nicht. Der Kehrbruch einer beliebigen rationalen Zahl ist sein inverses Element.
\end{example}

\theoremstyle{example}
\begin{example}{}
Da $\mathbb{Z}, \, \mathbb{Q}\setminus\{0\} \subset \mathbb{R}$ sind, folgt insbesondere, dass $(\mathbb{R},+)$ und $(\mathbb{R}\setminus\{0\}, \, \cdot \,)$ abelsche Gruppen sind. 
\end{example}

Aus dem letzten Beispiel geht hervor, dass es einfacher wäre zwei verschiedene Verknüpfungen  $+$ und $\cdot$ auf einer Menge $K$ für die Addition bzw. auf der Menge $K\setminus\{0\}$ für die Multiplikation in eine  neue algebraische Struktur zusammenzufasssen. Diese nennen wir wie folgt:

\section{Körper}
%Quelle: https://www.youtube.com/watch?v=qpFyN7XkFEA&index=14&list=PLLTAHuUj-zHgrxnm5NRR_vXH-pJ9ZrrvD
\label{sec:Körper}
\theoremstyle{defintion}
\begin{definition}{\textbf{Körper}}
Ein Körper besteht aus zwei Mengen $K$ und $K\setminus\{0\}$ mit zwei Verknüpfungen
\begin{align*}
	+&: K \times K \rightarrow K
	\\ \cdot &: K \times K \rightarrow K
\end{align*}

für die gelten:
\begin{enumerate}
\item $(K,+)$ ist eine abelsche Gruppe mit $e=0$ als das neutrale Element.
\item $(K\setminus\{0\}, \cdot)$ ist eine kommutative Gruppe mit dem Einselement $e=1$ als das neutrale Element.
\item(Distrivbutivgesetze)
Für alle $a, b, c \in K$ gilt
\begin{align*}
a \cdot (b+c)  &= a \cdot b + a \cdot c
\\ (a+b) \cdot c &= a\cdot c + b \cdot c \text{\, . (vgl. \cite{Körper})}
\end{align*}
\end{enumerate}
\end{definition}

\theoremstyle{example}
\begin{example}
Die Menge der rationalen Zahlen $\mathbb{Q}$ und der reellen Zahlen $\mathbb{R}$ bilden mit der uns geläufigen Addition und Multiplikation offensichtlich\footnote{Der Beweis ist nicht Gegenstand dieser Seminararbeit.} einen Körper. (vgl. \cite[S. 26]{Beutel})
\end{example}


\chapter{Vektorräume}
\label{sec:VR}

\section{Vektorräume}
\label{sec:Vektorräume}
\theoremstyle{definition}
\begin{definition}{\textbf{Vektorraum}}
\label{VR-axiome}
\\{\glqq}Sei $K$ ein Körper. Ein Vektorraum[\footnote{Über den Vektoren befinden sich in der höheren Mathematik keine Pfeile mehr und man notiert sie, um Platz zu sparen, meist waagrecht.}] über K (oder $K$-Vektorraum) ist eine kommutative Gruppe $(V,+)$[\footnote{Vektoraddition}] zusammen mit einer äußeren Verknüpfung 
	\begin{align*}
	\cdot : K \times V &\rightarrow V
	\\ (\lambda,v) &\mapsto \lambda \cdot v
	\end{align*}
(genannt Skalarmultiplikation) mit folgenden Eigenschaften: Für alle $\lambda, \lambda_1, \lambda_2 \in K$ und $v, v_1, v_2 \in V$ gilt
	\begin{enumerate}
		\item $(\lambda_1 + \lambda_2) \cdot v = \lambda_1 \cdot v + 	\lambda_2 \cdot v $
		\item $ \lambda \cdot (v_1 + v_2) = \lambda \cdot v_1 + \lambda \cdot v_2$
		\item $\lambda_1 \cdot (\lambda_2 \cdot v) = (\lambda_1 \cdot \lambda_2) \cdot v$
		\item $ 1 \cdot v = v$.{\grqq} \cite[S. 28, 6.1]{Skript}
	\end{enumerate}
\end{definition}

\theoremstyle{bem}
\begin{bem}{}
Vektorräume beinhalten zwei verschiedene Additionen und Multiplikationen, die allerdings aus Gründen der Übersichtlichkeit nicht extra gekennzeichnet werden. Zum einen gibt es die Addition in $K$ und in $V$, ebenso die Multiplikation in $K$, zum anderen die sogenannte Skalarmultiplikation\footnote{Wir kennen sie unter der Streckung oder Stauchung eines Vektors mit einem Skalar.} \(\lambda \cdot v \) für $\lambda \in K$ und $v \in V$. (vgl. \cite[S. 28, 6.1]{Skript})
\end{bem}

\theoremstyle{definition}
\begin{definition}{\textbf{Mehrfaches kartesisches Produkt}}
\label{exampleVR}
	\\Das mehrfache kartesische Produkt einer Menge $K$ wird wie folgt definiert:
\[\underbrace{K \times K \times ... \times K}_{n\text{-mal}} := K^n \text{. (vgl. \cite[S. 28]{Enzy})}\]
\end{definition}

\theoremstyle{example}
\begin{example}{$n$-faches kartesisches Produkt als $K$-Vektorraum}
Sei $K$ ein Körper. Für $n \in \mathbb{N}$ machen wir \[\text{{\glqq}} K^n = \{(x_1,...,x_n) \mid x_1,...,x_n \in K\} \text{{\grqq}} \, \text{\cite[S. 28, 6.2]{Skript}}\]zu einem $K$-Vektorraum mit folgender Definition von komponentenweisen Addition
\[\text{{\glqq}}(x_1,...,x_n)+(y_1,...,y_n) := (x_1 + y_1,x_2 + y_2,..., x_n + y_n)\text{{\grqq}}\, \text{\cite[S. 28, 6.2]{Skript}}\]
und Skalarmultiplikation
\[\text{{\glqq}}\lambda \cdot (x_1,...,x_n):= (\lambda x_1,\lambda x_2,...,\lambda x_n)\text{{\grqq}} \, \text{\cite[S. 28, 6.2]{Skript}.}\]
\end{example}

\theoremstyle{example}
\begin{example}{Ebene und Raum} \label{KoS}
\begin{figure}[h]
\begin{minipage}[t]{.5\textwidth}
\begin{tikzpicture}[scale=2]
   % Draw axes
   \draw [<->,thick] (0,2) node (yaxis) [above] {$y$}
       |- (2,0) node (xaxis) [right] {$x$};
   \coordinate (1) at (1,1);
   % Draw lines indicating intersection with y and x axis. Here we use
   % the perpendicular coordinate system
   \draw[dashed] (yaxis |- 1) node[left] {$y'$}
       -| (xaxis -| 1) node[below] {$x'$};
   % Draw a dot to indicate intersection point
   \fill[red] (1) circle (2pt);
      \draw (1,1.2) node {$(x',y')$};
	
\end{tikzpicture}
\captionsetup{singlelinecheck=off}
\caption{$\mathbb{R}^2=\mathbb{R} \times \mathbb{R}$}
\end{minipage}
\hfill
\begin{minipage}[t]{.45\textwidth}
\begin{tikzpicture} [scale=3]
    \draw [->] (0,0) -- (1,0,0) node [right] {$x_2$};
    \draw [->] (0,0) -- (0,1,0) node [above] {$x_3$};
    \draw [->] (0,0) -- (0,0,1.5) node [below left] {$x_1$};
    \coordinate (2) at (1,1,1);
    \fill[red] (2) circle (1pt);
    \draw[red] (0,0,0) -- (0,0,1) -- (1,0,1) -- (1,1,1);
\end{tikzpicture}
\caption{$\mathbb{R}^3=\mathbb{R} \times \mathbb{R} \times \mathbb{R}$}
\end{minipage}
\end{figure}
Das uns bekannte zweidimensionale Koordinatensystem $\mathbb{R}^2$ ist das kartesische Produkt aus \(\mathbb{R} \times \mathbb{R}\). Jeder Punkt der $x$-$y$-Ebene lässt sich durch ein {\glqq}Paar von Koordinaten\footnote{Alternativ nennen wir es auch Tupel.}, $(x, y)$,{\grqq} \cite[S. 28]{Enzy} für alle ${x, y\in \mathbb{R}}$ beschreiben. Das dreidimensionale Koordinatensystem $\mathbb{R}^3$ ist das dreifache kartesische Produkt der Menge der reellen Zahlen. Die Koordinaten werden als {\glqq}Tripel{\grqq}  \cite[S. 28]{Enzy} $(x_1, x_2, x_3)$ für alle $x_1, x_2, x_3\in \mathbb{R}$ bezeichnet.
\end{example}
%Enzy zitieren

\theoremstyle{definition}
\begin{definition}{Untervektorraum}
\label{UVR}
\\ \glqq Wir nennen eine nichtleere Teilmenge $U \subset V$ \acl{UVR}, falls gilt 
\begin{enumerate}
\item  $0_V$ \footnote{\label{foot:8}Den Nullvektor kennzeichnen in diesem Fall wir mit $0_V$. Sonst schreiben wir eine $0$.} $\in U$.
\item $u,v \in U \Rightarrow u + v \in U$ (Abgeschlossenheit bzgl. +)
\item $u \in U \text{,} \, \lambda \in K \Rightarrow \lambda u \in U$ (Abgeschlossenheit bzgl. Skalarmultiplikation $\cdot$\,).\grqq
\end{enumerate} 
\end{definition}
Zitat aus S.298 Tut.
In Abb. \ref{UVR} sehen wir einen eindimensionalen \acl{UVR} des \acl{R2}. Ein abstrakteres Beispiel für einen \acl{UVR} sehen wir in Abschnitt \ref{sec:RX}

Der $\mathbb{R}^2$ und $\mathbb{R}^3$ (vgl. \ref{KoS}) kann mit Hilfe eines kartesischen Koordinatensystems sehr gut visualisiert werden. Funktioniert das beispielsweise auch für den $\mathbb{R}^4$, den $\mathbb{R}^5$ oder sogar den $\mathbb{R}^n$ ? Es fehlt uns lediglich die Vorstellungskraft über die übrigen Richtungen im größer-drei-dimensionalen Raum. In diesem Kapitel werden wir den Begriff der \acl{Dim} definieren.
Auch lernen wir eine Möglichkeit, uns höherdimensionale Vektorräume \glqq formal\grqq \, vorzustellen.

\section{\aclp{EZS}}
\label{Erz}

\begin{figure}[h]
\centering
\def\svgwidth{150pt}
\input{Bilder/Linkomb.pdf_tex}
\caption{\(\sum \limits_{i=1}^{2} \lambda_i v_i = \lambda_1 v_1 + \lambda_2 v_2\), wobei hier $\lambda_1=\lambda_2=3$ ist}
\label{linkombgrafik}
\end{figure}

Aus dem Schulunterricht kennen wir bereits den Begriff der Linearkombination, eine Summe aus mehreren gestreckt und gestauchten Vektoren, woraus wiederum ein neuer Vektor entsteht. Im Folgenden wird dieser Sachverhalt auf eine allgemeinere mathematische Ebene gebracht, sodass es für beliebig viele Vektoren aus jedem \acl{VR} und für beliebig viele Skalare aus jedem Körper gilt.

\theoremstyle{definition}
\begin{definition}{\textbf{\acl{Linkomb}}}
\\Sei $K$ ein Körper, $V$ ein $K$-\acl{VR} und ${v_1,...,v_k \in V}$. Ein Vektor $v \in V$ lässt sich mit $v_1,...,v_k$ und den Koeffizienten $\lambda_i \in K$ für $i=1,...,k$ und $k \in \mathbb{N}$ wir folgt \emph{linear kombinieren}:
\[v = \sum_{i=1}^k \lambda_i v_i = \lambda_1 v_1 + \lambda_2 v_2 + \lambda_3 v_3 + ... + \lambda_k v_k \text{. (vgl. \cite[S. 298, 16.3]{Tut})}\]
\end{definition}
%indir Zitat aus tutorium S.298 Def.16.3

\theoremstyle{definition}
\begin{definition}{\textbf{lineare Hülle}}
	\\ Sei $V$ ein Vektorraum über den Körper $K$ und $A \subset V$. Die Menge aller \aclp{Linkomb} 
\[\text{\glqq}\langle A \rangle_K := \{\sum \limits_{i=1}^{n} \lambda_i a_i \mid n \in \mathbb{N}, \lambda_i \in K, a_i \in X \,\text{für} \, i = 1,...,n \}\text{\grqq \cite[S. 30]{Bosch}}\]
bezeichnen wir als den,  die {\glqq}\emph{lineare Hülle}{\grqq} \cite[S. 298, 16.4]{Tut}, das "\emph{Erzeugnis}" \cite[S. 298, 16.4]{Tut} oder den "\emph{Span}" \cite[S. 298, 16.4]{Tut} von $X$. 
%indir Zit. aus Bosch S.30 
\label{linhuelledef}
\end{definition}

\begin{figure} [!tbp]
 	\centering
	\begin{minipage}[b]{0.4\textwidth}
		\def\svgwidth{175pt}
		\input{Bilder/LinHuelle.pdf_tex}
		\caption{$\langle v \rangle_\mathbb{R} \in \mathbb{R}^2$ ist eine Gerade.} 
		\label{fig:UVR}
	\end{minipage}
\hfill
	\begin{minipage}[b]{0.4\textwidth}
		\def\svgwidth{175pt}
		\input{Bilder/2dKoS.pdf_tex}
		\caption{\(E = \{e_1,e_2\} \subset \mathbb{R}^2 \Rightarrow \langle E \rangle_{\mathbb{R}} = \mathbb{R}^2\).}
	\end{minipage}
\end{figure} 
%Man kümmere sich um die Bildunterschrift
 
\theoremstyle{definition}
\begin{definition}{\textbf{\acl{EZS}}}
	\\"Eine Teilmenge $M$ eines $K$-Vektorraums $V$ heißt \acl{EZS} von $V$, wenn $V = \langle M \rangle_K$. $V$ heißt \emph{endlich erzeugt} [von $M$]\footnote{\label{foot:1}Sei $M = \{m_1, m_2, ..., m_n\}$ eine $n$-elementige Teilmenge von $V$ aus den Vektoren $m_1, m_2, ..., m_n$.}, wenn es ein endliches \acl{EZS} gibt." \cite[S. 39, 9.4]{Skript}
\end{definition}

%\theoremstyle{example}
%\begin{example}{Zahlenmengen als Vektorräume?}
%\end{example} 

\theoremstyle{definition}
\begin{definition}{\textbf{\acl{Linunab}}}
\label{def:linunab}
	\\ \glqq Sei $K$ ein Körper und $V$ ein $K$-\acl{VR}, $v_1, ..., v_k \in V$. Wie nennen das System $(v_1, ..., v_k)$ \emph{\acl{linunab}}, wenn gilt
	\[\sum \limits_{i=1}^{k} \lambda_i v_i = 0 \Longrightarrow \lambda_i = 0 \,[...]\, \text{[für alle]}\, i \text{.}\]
	Andernfalls heißt \((\acl{vs})\) \emph{\acl{linab}}.{\grqq} \cite[S. 298, 16.5]{Tut} 
\end{definition}

Sei eine Menge \acl{linunab} und sei dies zu überprüfen. So folgt ist die Gleichung in \ref{def:linunab} nur durch die {\glqq}trivial[e]{\grqq} \cite[S. 307, 16.5]{Tut} Lösung, also alle Koeffizienten $\lambda_i = 0$, lösbar.
Sei dagegeben eine Menge \acl{linab} und sei dies zu überprüfen, dann ist mindestens ein $\lambda_i\not= 0$.
An dieser Stelle sei für Beispiele auf diesen und diesen Abschnitt hingewiesen.


\section{Basis}
\label{sec:Basis}
%Einleitungssatz zu Basis? Mit Lemma von Zorn beginnen?? 

\theoremstyle{theo} \label{theo:Basis}
\begin{theo}{\textbf{Basis}}
\\ \glqq Sei $V$ ein $K$-\acl{VR}. Eine Teilmenge $B \subseteq V$ heißt Basis von $V$, wenn die folgenden äquivalenten Bedingungen gelten:
		\begin{enumerate}
		\item \label{Basis1} $B$ ist ein \emph{\acl{EZS} \emph{und} \acl{linunab}}.
		\item \label{Basis2}$B$ ist [ein] \emph{minimales\footnote{Diese Eigenschaft gilt nicht mehr, wenn ein Element aus $B$ entfernt wird.} \acl{EZS}}.
		\item \label{Basis3}$B$ ist [eine] \emph{maximale\footnote{Wird ein $v \in V$ zu $B$ hinzugefügt, so ist $B$ nicht mehr \acl{linunab}.} \acl{linunab}e Teilmenge}.{\grqq} \cite[S. 41, 9.16]{Skript}
		\end{enumerate}
\end{theo}

Um das Theorem bestehend aus der Äquivalenzaussage (i) $\Leftrightarrow$ (ii) $\Leftrightarrow$ (iii) zu zeigen, reicht es den Beweis des sogenannten Ringschlusses für die Implikationen (i) $\Rightarrow$ (ii), (ii) $\Rightarrow$ (iii) und (iii) $\Rightarrow$ (i) zu führen.
\\ Im folgenden erfolgen die Beweise alle durch Widerspruch. Um oftmals die Inkonsistenzen zu finden, bedarf es einer gründlichen Arbeit mit den Voraussetzungen, welche zur Vereinfachung gekennzeichnet werden.

\begin{proof}  
Vor.: Sei $B = \{\acl{vs}\}$. $B$ ist \acl{EZS} und \textbf{\acl{linunab}}.
\\ Beh.: (i) $\Rightarrow$ (ii) 
\\ Bew.: \acl{ang} $B$ ist nicht minimal. Dann ist $B$ ohne $v_k$ für $k=1,...,n$ immer noch ein \acl{EZS}:
\\ Sei $\lambda_i \in K$ für $i=1,...,n$.
\begin{align*}\Rightarrow v_k = \sum \limits_{\substack{i=1\\i\not=k}}^{n}\lambda_i v_i
\Leftrightarrow 0 = \sum \limits_{\substack{i=1\\i\not=k}}^{n}\lambda_i v_i - v_k 
\end{align*}
\(\Rightarrow \lambda_k \not= 0 \Rightarrow \text{\textbf{linear abhängig}} \, \lightning
\Rightarrow \text{Beh.} \)
\\In Wortform: Der Nullvektor erhält somit eine nichttriviale Darstellung, da $\lambda_k=1\not=0$. Insbesondere ist das System $B$ mit $v_k$ für $k=1,...,n$ \textbf{linear abhängig}, was ein Widerspruch zu der Voraussetzung der linearen Unabhängigkeit ist, woraus folgt, dass die Behauptung wahr ist. (vgl. \cite[S. 41, 9.16 (a) $\Rightarrow$ (b)]{Skript})
\\
\\ Vor.: B ist ein \textbf{minimales \text{\acl{EZS}}}.
\\ Beh.: (ii) $\Rightarrow$ (iii) 
\\ Bew.: Zuerst zeigen wir, dass $B$ \acl{linunab} ist.
\\ Angenommen $B$ ist \text{\acl{linab}}. Sei \(\lambda_i \, \text{für} \, i=1,...,n\). Dann hat die Gleichung \(\acl{Summe}=0\) mindestens ein $\lambda_i\not=0$ mit $i=1,...,n$ als Lösung.
\\Wir nehmen o.B.d.A\footnote{Ohne Beschränkung der Algmeinheit. Wir schauen uns einfachheitshalber einen Spezialfall an, um die Aussage im Allgemeinen zu beweisen. Die anderen Fälle würden analog zu zeigen sein.} an:
Sei $\lambda_j \not= 0$ für $j \in {1,...,n}$. $B$ ist linear abhängig aufgrund $v_j$. 
\begin{align*}
&\Rightarrow \lambda_j v_j + \sum\limits_{\substack{i=1\\i\not=j}}^{n} \lambda_i v_i = 0
\\ &\Leftrightarrow \lambda_j v_j = - \sum\limits_{\substack{i=1\\i\not=j}}^{n} \lambda_i v_i
\\ &\Leftrightarrow v_j = - \sum\limits_{\substack{i=1\\i\not=j}}^{n} \frac{\lambda_i}{\lambda_j} v_i \in\, \langle B\, \backslash \{v_j\}\rangle_K
\end{align*}
Der Vektor $v_j$ liegt in $B \, \backslash \, \{v_j\}$. Wir benötigen, um den Widerspruch zu erhalten, einen Zwischenbeweis, der aussagt, dass die Menge $B \, \backslash \, \{v_j\}$ entgegen der Minimalität ein \acl{EZS} ist.
\par
\begingroup
\leftskip=2cm
\noindent
\\Zw.vor.: \(v_j \in B \, \backslash \, \{v_j\}\text{, also} \, v_j \cup B \, \backslash \, \{v_j\} \, \text{linear abhängig.} \)
\\Zw.beh.:$B \, \backslash \, \{v_j\}$ ist ein \acl{EZS}
\\Zw.bew.: Angenommen, $B \, \backslash \, \{v_j\}$ sei kein \acl{EZS}. Dann gibt es ein $v_j \notin B \, \backslash \, \{v_j\}$ für $k\in\{1,...,n\}$. Daraus folgt allerdings: Die Menge $B \, \backslash \, \{v_j\} \cup v_j = B \cup {v_j} $ ist \acl{linunab}, was der Voraussetzung, dass $B \cup {v_j}$ \acl{linab} ist, widerspricht.\footnote{Zw.bew. ist eigens vom Verfasser geführt worden.}
\\ 
\par
\endgroup
Daher ist $B \, \backslash \, \{v_j\}$ ein \acl{EZS}, was im Gegensatz zu der \textbf{Minimalität des \acl{EZS}s} von $B$ steht. Also ist $B$ linear unabhängig.
\\Es ist noch zu zeigen , dass $B$ \emph{maximal} \acl{linunab} ist. 
\\Angenommen, $B$ ist keine maximal \acl{linunab}e Teilmenge.
Dann gibt es ein $v \in V$, sodass \( B \cup \{v\}\,\text{\acl{linunab} ist}\)\, \textendash \, $v$ wäre nicht im Span von $B$\, \textendash, aber $B$ \textbf{erzeugt} laut Voraussetzung alle Vektoren in $V$, somit gilt $v \in \langle B \rangle$, was der Widerspruch ist. Daraus folgt, dass $B$ \emph{maximal} \acl{linunab} ist. (vgl. \cite[S. 41, 9.16 (b) $\Rightarrow$ (c)]{Skript})
\\
\\ Vor.: B ist ein \text{maximale} \acl{linunab}e Teilmenge.
\\ Beh.: (iii) $\Rightarrow$ (i) 
\\ Bew.: Es sind zwei Tatsachen zu zeigen, zum einen dass aus (iii) die Lineare Unabhängigkeit von $B$ folgt, was offensichtlich aus der Voraussetzung folgt, und zum anderen dass $B$ ein \acl{EZS} ist. 
\\ Angenommen, $B$ ist kein \acl{EZS}. Dann gibt es ein $v \in V$, sodass $v \notin \langle B \rangle$ gilt, also nicht im Erzeugnis von $B$ liegt. Daraus folgt, dass $\{v\} \cup B$ linear unabhängig ist, was der \textbf{Maximalität} von $B$ widerspricht. (vgl. \cite[S. 59]{Beutel}) 
\end{proof}

\theoremstyle{definition}
\begin{definition}{\textbf{Dimension eines \acl{VR}s}}
\label{def:dim}
	\\"Ist $B$ eine Basis eines endlich erzeugten \acl{VR}s $V$, so nennt man $n := |B| \in \mathbb{N}_0$ die Dimension von $V$. Wir schreiben dafür $\dim(V)= n$. Ist $V$ nicht endlich erzeugt, so setzen wir $\dim(V):= \infty$." \cite[S. 504]{Enzy}
\end{definition}

\theoremstyle{definition}
	\label{def:Ev}
	\begin{definition}{\textbf{\aclp{Ev}}}
	\\"Dabei sei $e_i \in K^n$ für $i = 1,...,n$ der $i$-te \acl{Ev}, also 
	\[e_i = (0,...,0,1,0,...,0)\text{,}\]
	wobei die 1 genau an der $i$-ten Stelle steht. Auf präzisere Weise können wir
	\[\text{[...]} \, [e_i = (\delta_{1i},\delta_{2i},...,\delta_{ii},...,\delta_{ni})]\]%Soll ich direkt oder inidir zitieren???? Bosch S.31
	schreiben mit 
	\[\delta_{hi}= \begin{cases} 1 \, \text{für} \: h = i \\ 0 \: \text{[...] [für} \, h\not=i] \end{cases}\]
	$\delta_{hi}$ ist das so genannte \emph{Kronecker-Symbol}.[\footnote{Nur wenn beide Indizes gleich sind ,dann $\delta_{ii}=1$}]" \cite[S. 31]{Bosch}
	\end{definition} 

\theoremstyle{example}
\begin{example}{Einheitsvektoren}
\\ In der Abbildung \ref{UVR} siehen wir die \aclp{Ev} des \acl{R2}, die zudem auch eine Basis bilden. Nun werden wir die Standardbasisvektoren für den $K^n$-\acl{VR} für jedes $n \in \mathbb{N}$ kennenlernen.
Es ist noch zu zeigen, dass die \aclp{Ev} eine Basis bilden. Hierzu nutzen wir die Eigenschaften aus (\ref{Basis1}) des Theorems \ref{theo:Basis}:
"Es ist $\{e_1,...,e_n\}$ ein \acl{EZS} von $K^n$ (denn für $v =(\lambda_1,...,\lambda_n) \in K^n$ ist $v = \sum \limits_{i=1}^{n} \lambda_i e_i)$.\grqq" \cite[S. 40, 9.9]{Skript}
%Erklärung??
\\"In $K^n$ sind die \aclp{Ev} $e_1,...,e_n$ \acl{linunab}: \[0=\sum \limits_{i=1}^{n} \lambda_i e_i = (\lambda_1,...,\lambda_n)\text{.}\]
Aus der Definition \ref{def:Ev} folgt: $\lambda_1=...=\lambda_n=0$." \cite[S. 41, 9.14]{S. 41}
%Erklärung!!! Skript 9.14
\\ Somit ist ${e_1,...,e_n}$ eine Basis von $K^n$. Wir bezeichnen sie auch als "kanonische Basis" \cite[S. 42, 9.17]{Skript} . Daraus folgt, dass "$dim(K^n) = n$" \cite[S. 44, 9.24 (a)]{Skript} ist.
%Letzter satz aus Skript 9.23 Beispiel (b)
%Skript 9.17
\end{example}


\chapter{unendlichdimensionale Vektorräume}
\label{sec:unendliche VRs}

\section{Polynomvektorräume}
$\infty$ $\infty$ $\infty$ $\infty$ $\infty$ $\infty$ $\infty$

\section{Vektorraum der unendlichen Folgen}

\chapter{Schluss}
\label{sec:Schluss}
\section{Zusammenfassung}
Im Laufe der Arbeit haben wir sowohl die visuelle als auch die mathematische Darstellung endlicher und unendlicher Dimension kennen gelernt. Begonnen haben wir mit Gedankenexperimenten zur Visualisierung von $n$-dimensionalen Koordinatensystemen und Würfeln (Kapitel \ref{sec:Visualisierung}). Anschließend haben wir durch die grundlegenden Kenntnisse der linearen Algebra (Kapitel \ref{sec:LA}) über Gruppen und Körper, \aclp{VR}, lineare (Un-)Abhängigkeit sowie \aclp{EZS}n die Basis definiert, deren Mächtigkeit die Dimension eines \acl{VR}s ist. Zum Schluss haben wir uns mit Räumen unendlicher Dimension wie dem $\mathfrak{c}^{\mathfrak{c}}$-dimensionalen Funktionenraum $F$ aus der Menge aller reellen Abbildungen von $\mathbb{R}\rightarrow\mathbb{R}$ und seinem \acl{UVR}, dem $\aleph_{0}$-dimensionalen Polynomvektorraum $\acl{RX}$, befasst. 
%%%%%%%%%%%%%%%%%%%%%%%%%%%%%%%%%%%%%%%%%%%%%%%%%%%%%%%%%%%%%%%%%%%%%
\section{Ausblick}
Dass jeder \acl{VR}, sei es von endlicher oder unendlicher Dimension, eine Basis hat (Zornsches Lemma), und Satz \ref{dim} wurden in dieser Arbeit nicht bewiesen, da dazu weitreichende Kenntnisse aus der Mengenlehre benötigt werden. In der Einleitung haben wir Würfel in $n$ Dimensionen betrachtet. Man könnte weitere platonische Körper wie Tetraeder, Oktaeder, Dodekaeder und Ikosaeder in höheren Dimensionen behandeln. Es gibt einen anderen Basenbegriff, der im Gegensatz zur \textsc{Hamel}-Basis (Theorem \ref{theo:Basis}) unendliche Summen zulässt. So kann man eine überabazählbare Menge zu eine einer "handhabbaren, abzählbaren" \cite[S.762]{Springer}  machen.
\begin{definition}\cite[S.762, 7.67]{Springer} (\emph{Schauderbasis}) \glqq Eine Folge $v_1,v_2,...$ in $V$ heißt \textsc{Schauder}-Basis von $V$, wenn gilt:
Zu jedem $v \in V$ gibt es eindeutige $\alpha_i \in K$, $i \in \mathbb{N}$[,] so dass \[v = \sum \limits_{n=1}^{\infty} \alpha_n v_n \;\text{gilt.\grqq}\]
\end{definition}
Wir haben uns bisher nur mit der mathematischen Theorie beschäftigt, ohne auf ihre Anwendung einzugehen. Man findet sie heutzutage überall wie zum Beispiel in der Informatik, Physik, Stochastik und Datenanalyse. Keine dieser Wissenschaften kann das in dieser Arbeit erläuterte Fundament entbehren. 
%Literaturverzeichnis
\bibliographystyle{unsrtdin}
\bibliography{Literatur}


\end{document}
