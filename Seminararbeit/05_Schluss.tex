\chapter{Schluss}
\label{sec:Schluss}
\section{Zusammenfassung}
Im Laufe der Arbeit haben wir sowohl die visuelle als auch die mathematische Darstellung endlicher und unendlicher Dimension kennen gelernt. Hierbei haben wir insbesondere darauf acht gelegt, die Aussagen möglichst allgemein zu formulieren. Begonnen haben wir mit Gedankenexperimenten zur Visualisierung von $n$-dimensionalen Würfeln und Koordinatensystemen (Kapitel \ref{sec:Visualisierung}). Anschließend haben wir durch die grundlegenden Kenntnisse der linearen Algebra über Gruppen und Körper (Kapitel \ref{sec:Algebraische Strukturen}), \aclp{VR}, lineare (Un-)Abhängigkeit sowie \aclp{EZS}n die Basis definiert, dessen Mächtigkeit die Dimension eines \acl{VR}s ist (Kapitel  \ref{sec:VR}). Zum Schluss haben wir uns mit Räumen unendlicher Dimension wie dem Funktionenraum der reellen Abbildungen befasst. Die überabzählbare Dimension dieses \acl{VR}s $c^c$ wird durch einen Satz bewiesen, der besagt, wenn die Mächtigkeit eines \acl{VR}s (Anzahl der Linearkombinationen) größer als die seines zugehörigen Körpers ist, dass $\dim(V)=|V|$ gilt. In diesem Fall gibt es $c^c$ Abbildungen von $\mathbb{R}\rightarrow\mathbb{R}$ bzw. Elemente des Funktionenraumes. Der Polynomvektorraum {\textendash} ein \acl{UVR} des Raumes {\textendash} besitzt hingegen eine abzählbare Dimensionenzahl $\aleph_0$, da sich eine bijektive Funktion von den Basiselementen, den Monomen, zu den natürlichen Zahlen findet.
\section{Weiterführung}

Dass jeder \acl{VR}, sei es von endlicher oder unendlicher Dimension, eine Basis hat (Zornsches Lemma), wurde in dieser Arbeit nicht bewiesen, da dazu Kenntnisse aus der Mengenlehre benötigt werden.
\\ Es gibt einen anderen Basenbegriff, der im Gegensatz zur Hamelbasis (Theorem \ref{theo:Basis}) unendliche Summen zulässt. So kann man eine überbazählbare Menge zu eine einer "handhabbaren, abzählbaren" erhalten erhalten.

\begin{definition}{\textbf{Schauderbasis}}
\\ Eine Folge $v_1,v_2,...$ in $V$ heißt \textsc{Schauder}-Basis von $V$, wenn gilt:
Zu jedem $v \in V$ gibt es eindeutige $\alpha_i \in K$, $i \in \mathbb{N}$, so dass 
\begin{align*}
v = \sum \limits_{n=1}^{\infty} \alpha_n v_n\text{.}
\end{align*}
\end{definition}
\cite[S. 762, 7.67]{Spring} 

Mit Vektorräumen solcher Art beschäftigt man sich in der Funktionalanalysis, ein Teilgebiet der Mathematik, der mit unendlichdimensionalen \aclp{VR} arbeitet.
\\ In der Einleitung haben wir Würfel in $n$ Dimensionen betrachtet. Man könnte weitere platonische Körper (Tetraeder, Oktaeder, Dodekaeder, Ikosaeder) in höheren Dimensionen in einer weiteren Arbeit behandeln. 
\\ Wir haben uns bisher nur mit der mathematischen Theorie beschäftigt, ohne auf ihre Anwendung einzugehen. Man findet sie heutzutage überall zum Beispiel in der Informatik, Physik, Stochastik und Datenanalyse. Keine dieser Wissenschaften kann das in dieser Arbeit erläuterten Fundament entbehren. 