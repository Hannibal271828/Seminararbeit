\chapter{Schluss}
\label{sec:Schluss}
\section{Zusammenfassung}
Im Laufe der Arbeit haben wir sowohl die visuelle als auch die mathematische Darstellung endlicher und unendlicher Dimension kennen gelernt. Begonnen haben wir mit Gedankenexperimenten zur Visualisierung von $n$-dimensionalen Koordinatensystemen und Würfeln (Kapitel \ref{sec:Visualisierung}). Anschließend haben wir durch die grundlegenden Kenntnisse der linearen Algebra (Kapitel \ref{sec:LA}) über Gruppen und Körper, \aclp{VR}, lineare (Un-)Abhängigkeit sowie \aclp{EZS}n die Basis definiert, deren Mächtigkeit die Dimension eines \acl{VR}s ist. Zum Schluss haben wir uns mit Räumen unendlicher Dimension wie dem $\mathfrak{c}^{\mathfrak{c}}$-dimensionalen Funktionenraum $F$ aus der Menge aller reellen Abbildungen von $\mathbb{R}\rightarrow\mathbb{R}$ und seinem \acl{UVR}, dem $\aleph_{0}$-dimensionalen Polynomvektorraum $\acl{RX}$, befasst. 
%%%%%%%%%%%%%%%%%%%%%%%%%%%%%%%%%%%%%%%%%%%%%%%%%%%%%%%%%%%%%%%%%%%%%
\section{Ausblick}
Dass jeder \acl{VR}, sei es von endlicher oder unendlicher Dimension, eine Basis hat (Zornsches Lemma), und Satz \ref{dim} wurden in dieser Arbeit nicht bewiesen, da dazu weitreichende Kenntnisse aus der Mengenlehre benötigt werden. In der Einleitung haben wir Würfel in $n$ Dimensionen betrachtet. Man könnte weitere platonische Körper wie Tetraeder, Oktaeder, Dodekaeder und Ikosaeder in höheren Dimensionen behandeln. Es gibt einen anderen Basenbegriff, der im Gegensatz zur \textsc{Hamel}-Basis (Theorem \ref{theo:Basis}) unendliche Summen zulässt. So kann man eine überabazählbare Menge zu eine einer "handhabbaren, abzählbaren" \cite[S.762]{Springer}  machen.
\begin{definition}\cite[S.762, 7.67]{Springer} (\emph{Schauderbasis}) \glqq Eine Folge $v_1,v_2,...$ in $V$ heißt \textsc{Schauder}-Basis von $V$, wenn gilt:
Zu jedem $v \in V$ gibt es eindeutige $\alpha_i \in K$, $i \in \mathbb{N}$[,] so dass \[v = \sum \limits_{n=1}^{\infty} \alpha_n v_n \;\text{gilt.\grqq}\]
\end{definition}
Wir haben uns bisher nur mit der mathematischen Theorie beschäftigt, ohne auf ihre Anwendung einzugehen. Man findet sie heutzutage überall wie zum Beispiel in der Informatik, Physik, Stochastik und Datenanalyse. Keine dieser Wissenschaften kann das in dieser Arbeit erläuterte Fundament entbehren. 