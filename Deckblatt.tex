\documentclass[12pt,a4paper,twoside,openright]{scrreprt}
\usepackage{amsfonts}
\usepackage{amsmath}
	\DeclareMathOperator{\Abb}{Abb}
\usepackage{amsthm}
\usepackage{hyperref}
\usepackage{stmaryrd}
\usepackage{mathtools}

\begin{document}

\thispagestyle{empty}


\mdseries{

\begin{center}
{\Large
\textsc{Albrecht-Altdorfer-Gymnasium Regensburg}\\
Oberstufenjahrgang 2018-2019}\\
\rule{7cm}{1pt}

\vspace*{0.5cm}

\textsc{Seminararbeit}\\ 
Unendlichkeit in der Mathematik\\
Leitfach: Mathematik
\vspace*{2cm} 

{\LARGE
\textmd{Einblick in höhere Dimensionen}
}

\vspace*{1.5truecm} 



{\large
\vspace*{0.3cm}
Hanifah Muhammad
}

\end{center}

%\vspace*{1.0cm}

\begin{tabular}{@{}ll}
\small
\\[0.5cm]
\normalsize
\quad Kursleiterin: Sonja Bräu \\
\quad Abgabetermin: 6.11.2018 \\
\quad Abgegeben am 6.11.2018\\
\quad Abschlusspräsentaion am 15.11.2018\\
\small

\end{tabular}
\vspace*{0.5truecm}

    \begin{tabular}{| l | l | l | l | l | l |}
    \hline
    \textbf{Bewertung} & \textbf{Note} & \textbf{Notenstufe in Worten} & \textbf{Punkte} &   & \textbf{Gesamt} \\ \hline
    schriftliche Arbeit &  &  &  & $\times 3$ &   \\ \hline
    Abschlusspräsentation &  &  &  & $\times 1$ &  \\ \hline
    \end{tabular}  

\vspace*{0.5truecm}

\begin{center}
\begin{Huge}
\begin{tabular}{| l |p{2cm}|}
\hline
$\sum$ &  \\ \hline
\normalsize{Gesamtleistung nach § 61 (7) GSO: $\frac{\sum}{2}\approx$}&\\
\normalsize{Die doppelte Wertung (max. 30 Punkte) geht in die Gesamtqualifikation ein.}&\\ \hline
\end{tabular}
\end{Huge}
\end{center}

\vspace*{3truecm}

\rule{8cm}{0.4pt}\\
Datum und Unterschrift der Kursleiterin

\clearpage
}
\end{document}